\textcolor{blue}{
Before presenting the theorem-proof, we outline the following points:
\begin{itemize}
    \item If domain $[0,1]^n$ is compact and $f$ is continuous on this compact set then $f$ is uniformly continuous.
    \item Multivariate functions can be decomposed into sums of univariate functions.
\end{itemize}
\begin{proof}
We will construct the functions $\phi_{p,q}$ and $\Phi_p$ step by step, using the fact that a continuous function on a compact domain can be approximated by a suitable combination of univariate functions. The goal is to approximate $f$ using univariate functions $\phi_{p,q}$ (Inner Functions) and a continuous aggregation function $\Phi_p$ (Outer Functions).
A. \textbf{Constructing $\phi_{p,q}$}(Inner Functions)
Kolmogorov introduced a specific mapping of the domain into 1D intervals. For example, divide the domain $[0, 1]^n$ into slices along one coordinate, say $x_1$, and map the slices to a function $\phi_{1,1}(x_1)$. Repeat for $x_2,x_3,\ldots,x_n$ to construct a set of univariate functions $\phi_{p,q}(x_q)$ for all $p$ and $q$.
B. \textbf{Constructing $\Phi_p$}(Outer Functions)
The outer function $\Phi_p$ aggregates the sums of the inner functions $\phi_{p,q}$. Specifically, for the given function $f(x_1, \ldots,x_n)$, we define $\Phi_p$ as a reconstruction rule based on the combined outputs of $\phi_{p,q}$
4. \textbf{Proving Continuity}
By uniform continuity of $f$, the functions $\phi_{p,q}$ and $\Phi_p$
can be chosen to ensure that their composition is continuous. Specifically, ensure that small changes in inputs $x_q$ lead to small changes in $\phi_{p,q}(x_q)$ and $\Phi_{p}(\textbf{x})$, preserving the continuity of $f$.
5. \textbf{Ensuring Uniqueness of Representation}
Kolmogorov showed that at most $2n+1$ inner functions are required. The choice of $\phi_{p,q}$ and $\Phi_{p}$is not unique, but the representation always exists
\end{proof}
}

%%%%%%%%%%%%%%%%%%%%%%%%%%%%%%%%%%%%%

\subsection{B-splines}
B-splines are essentially curves made up of polynomial segments, each with a specified level of smoothness. Picture each segment as a small curve, where multiple control points influence the shape. Unlike simpler spline curves, which rely on only two control points per segment, B-splines use more, leading to smoother and more adaptable curves.

The main advantage of b-splines lies in their local impact. Adjusting one control point affects only the nearby section of the curve, leaving the rest undisturbed. This property offers remarkable advantages, especially in maintaining smoothness and facilitating differentiability, which is crucial for effective backpropagation during training.